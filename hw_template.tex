
% latexmk -lualatex -file-line-error -synctex=1 -shell-escape hw_template.tex

% preamble {{{1

\documentclass[letter]{article}

\usepackage[T1]{fontenc}

\usepackage[margin=1in]{geometry}
\usepackage[document]{ragged2e}
\usepackage{graphicx}
\usepackage[en-GB]{datetime2}
\usepackage{enumitem}
\usepackage{dsfont}
\usepackage{ragged2e}
\usepackage{scalerel}
\usepackage{calc}
\usepackage{mdframed}
\usepackage{emoji}
\mdfdefinestyle{FullW}{%
    innertopmargin=\baselineskip,
    innerbottommargin=\baselineskip,
    userdefinedwidth=\textwidth}
\usepackage{array}
\newcolumntype{L}{>{$}l<{$}}

\usepackage{float}
\usepackage{amsmath}
\usepackage{amsthm}
\usepackage{amssymb}
\theoremstyle{definition}
\newtheorem*{theorem}{Theorem}

\usepackage{multicol}

\makeatletter

\newcommand*\wordpar[2]{{#1}\parbox[t]{\linewidth - \widthof{#1}}{#2}}

\newcommand*\question[1]{%
  \noindent\rule{\textwidth}{0.4pt}
  \Large
  \textbf{Question #1.} \\\vspace{-9pt}%
  \noindent\rule{\textwidth}{0.4pt}\vspace{-20pt}
}

\makeatother

% just for this example
\newcommand*\tbs[0]{\textbackslash}
% just for this example

\title{Homework 0}
\author{Dylan McClure}
\date{\today}

% }}}

\begin{document}

\begin{titlepage} % title page {{{1
  \Large
  \begin{flushright}
    MGF--3301 \\
    Fall 2021 \\
    \DTMdisplaydate{2021}{9}{28}{-1}
  \end{flushright}
  \begin{center}
    \vspace{0.4in} \textsc{\Huge Florida State University} \\
    \vspace{0.4in} \includegraphics[scale=0.75]{fsu_logo.png} \\
    \vspace{0.4in} \rule[0.2cm]{13cm}{0.1cm} \\ \vspace{0.4cm}
    {\Huge \bfseries Homework 0} \\[0.4cm]
    {\LARGE \bfseries Sundstrom Section 0.0}{\Large \quad \# 0, -1, -2} \\
    {\LARGE \bfseries Sundstrom Section 0.1}{\Large \quad \# -$\infty$, $\infty$ } \\
    \vspace{0.4in} \rule[0.2cm]{13cm}{0.1cm} \\
    \vspace*{\fill}
    Dylan McClure \emoji{rose}
  \end{center}
\end{titlepage} % }}}

\section*{\textsection0.0}

\question{0}

\textbf{Proposition:} Tacos are tasty.

\begin{proof}
  Let's assume tacos are tasty. Therefore, $\exists t$ where $t$ is a taco, and
  $$
  \left( t^t \rightarrow \neg\frac{t}{\sqrt{t}} \right) \equiv \;\mathrm{tacos}.
  $$
  As such we have proved that tacos are tasty.
\end{proof}

\begin{enumerate}[nosep]

  \item \begin{enumerate}[nosep,label=(\alph*)]
      \item I also use this
      \item for enumerations
    \end{enumerate}

  \item \begin{enumerate}[nosep,label=(\alph*)]
    \item because it looks
    \item pwetty
  \end{enumerate}

  \item \begin{enumerate}[nosep,label=(\alph*)]
    \item and yes,
    \item they're just snippets \emoji{eyes}
  \end{enumerate}
\end{enumerate}

\subsection*{Math}

\textbf{Inline Math Mode:} $ax = (b^2 + b)x$


\textbf{Display Math Mode:}
$$
ax = (b^2 + b)x
$$

\textbf{Multi-line Display Math:}
\begin{gather*}
  ax = (b^2 + b)x \\
     = b^2x + bx
\end{gather*}

\textbf{Align Parts (Multi-line):}
\begin{align*}
  ax &= (b^2 + b)x \\
     &= b^2x + bx
\end{align*}

\textit{Note:} the \& aligns the equations.

\textbf{Tables:}

\begin{table}[H]
  \center
  \begin{tabular}{|c|c|c|}
    \hline
    \textbf{col 1} & \textbf{col 2} & \textbf{col 3} \\ \hline\hline
    How & I & Make \\ \hline
    tables & \textit{woot} & \textit{woot} \\ \hline
  \end{tabular}
\end{table}

\subsection*{Useful Symbols}

\begin{table}[H]
  \center
  \begin{tabular}{r|l}
    \texttt{\$\tbs forall x \tbs in \tbs mathds\{Z\}\$} & $\forall x \in \mathds{Z}$ \\
    \texttt{\$\tbs exists x \tbs in \tbs mathds\{R\}\$} & $\exists x \in \mathds{R}$ \\
    \texttt{\$p \tbs rightarrow q\$}                    & $p \rightarrow q$     \\
    \texttt{\$p \tbs leftrightarrow q\$}                & $p \leftrightarrow q$ \\
    \texttt{\$\tbs phi \tbs Rightarrow \tbs rho\$}      & $\phi\Rightarrow\rho$ \\
    \texttt{\$\tbs left(x\^{}2\tbs right)\$}            & $\left(x^2\right)$    \\
    \texttt{\$\tbs neg\$}                               & $\neg$                \\
    \texttt{\$\tbs frac\{3\}\{4\}\$}                    & $\frac{3}{4}$         \\
    \texttt{\$\tbs sqrt\{ab\}\$}                        & $\sqrt{ab}$           \\
    \texttt{\$\tbs sqrt[3]\{ab\}\$}                     & $\sqrt[3]{ab}$        \\
    \texttt{\$a \tbs wedge b\$}                         & $a \wedge b$          \\
    \texttt{\$a \tbs vee b\$}                           & $a \vee b$            \\
  \end{tabular}
\end{table}

\end{document}

